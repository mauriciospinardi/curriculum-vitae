\iffalse
 * Author:          Maurício Spinardi
 * Creation date:   2019-01-14
 * File:            curriculum-vitae.tex
 * License:         LPPL 1.3c
 * Platform:        <not specified>
 * Project:         curriculum-vitae
 *
 * Description: Curriculum Vitae [en-us]
\fi

% Document class options are: ('10pt', '11pt' and '12pt'), ('a4paper',
% 'letterpaper', 'a5paper', 'legalpaper', 'executivepaper' and 'landscape'),
% ('sans' and 'roman')
\documentclass[11pt,a4paper,sans]{moderncv}

\usepackage[bottom=3cm,scale=0.75,top=3cm]{geometry}
\usepackage[utf8]{inputenc}
\usepackage{booktabs}
\usepackage{moderntimeline}
\usepackage{progressbar}
\usepackage{relsize}

% Document settings
% --------------------------------------

\moderncvcolor{blue}
\moderncvstyle{classic}
\renewcommand{\familydefault}{\sfdefault}

% Timeline settings
% --------------------------------------

\setlength{\hintscolumnwidth}{2.85cm}       % Length
\tlmaxdates{2011}{2020}                     % Scale
\tlwidth{0}

% Progress bar definitions
% --------------------------------------

\definecolor{pbblue}{rgb}{0.22,0.45,0.70}
\progressbarchange{filledcolor=pbblue,heightr=1,roundnessa=2pt,
width=.75\linewidth}

% Customized commands
% --------------------------------------

\newcommand\CC
{
    C\nolinebreak[4]\hspace{-.05em}\raisebox{.4ex}{\relsize{-3}{\textbf{++}}}
}

\newcommand\CF
{
\vfill
    \textcolor{gray}{
        Written in \LaTeX \protect \\ \\
        \small{
            \url{https://github.com/mauriciospinardi} \\
            \url{https://hackerrank.com/mauriciospinardi}
        }
    }
}

% Profile header
% --------------------------------------

\name{\Huge Maurício}{Spinardi}
\title{\Large Developer of embedded payment software solutions}
\address{}{Liberdade, 01514-000}{São Paulo, SP, Brasil}
\phone[mobile]{+55~(11)~98582~1184~[VIVO]}
\email{mauricio.spinardi@gmail.com}
\extrainfo{27 y/o, single}
\photo[64pt][0.4pt]{../misc/profile_pic}

% Document
% --------------------------------------

\begin{document}

\maketitle

\section{Academic background}
% --------------------------------------

\tlcventry{2017}{2019}
{M.Sc. in Computational Science}
{}{}{}{Federal University of São Carlos. Sorocaba, São Paulo.}

\tlcventry{2011}{2016}
{B.Sc. in Computer Science}
{}{}{}{Federal University of São Carlos. Sorocaba, São Paulo.}

\section{About me}
% --------------------------------------

\cvitem{}{Software developer, familiar with C, C++, Java and Lua. Experienced
with embedded payment solutions (ISO 8583, EMV contact and contactless card
processing) and on-demand maintenance. Currently working with hardware from
Verifone (Verix eVo and Engage V-OS2) and Ingenico manufactors (Telium 2 and
Telium Plus).}

\section{Professional background}
% --------------------------------------

\tlcventry{2017}{0}
{\textit{\small{(Current)}} System Software Analyst}
{SETIS Automação e Sistemas Ltda}{}{}
{Development and maintenance of embedded payment software solutions.}

\tldatecventry{2016}
{Intern}
{SETIS Automação e Sistemas Ltda}{}{}
{Professional initiation in the electronic card payment segment.}

\tldatecventry{2014}
{Teaching Assistant}
{Federal University of São Carlos}{Sorocaba}{São Paulo}
{Object Oriented Programming related tasks.}

\section{Languages}
% --------------------------------------

\cvitem{Português}{\textbf{Native}}
\cvitem{English}{\textbf{Intermediate \tiny{TOEFL®ITP: 613/677 (Exp. date: Oct.
2019)}}}
\cvitem{Español}{\textbf{Intermedio}}

\CF

% ------------------------------------------------------------------------------
\newpage

\section{Events and Certificates}
% --------------------------------------

\begin{minipage}[htb]{.5\textwidth}
    \cvitem{title}{\emph{Telium SDK Training}}
    \cvitem{minister}{Denise Tanizaka}
    \cvitem{institution}{Ingenico Group}
    \cvitem{date}{nov/2017}
\end{minipage}
\begin{minipage}[htb]{.5\textwidth}
    \cvitem{title}{\emph{Restrição = Inovação}}
    \cvitem{minister}{Fabio Akita}
    \cvitem{institution}{Federal University of São Carlos}
    \cvitem{date}{aug/2015}
\end{minipage}

\vspace{5mm}

\begin{minipage}[htb]{.5\textwidth}
    \cvitem{title}{\emph{Free Software and Your Freedom}}
    \cvitem{minister}{Richard M. Stallman}
    \cvitem{institution}{Federal University of São Carlos}
    \cvitem{date}{jan/2013}
\end{minipage}

\section{Technical skills}
% --------------------------------------

\vspace{1.5mm}

\cvitem{\textbf{Programming Languages}}
{
    \begin{tabular}{p{25mm}p{\linewidth}}
        \textbf{ C} & \progressbar{0.90} \protect\\
        \textbf{\CC} & \progressbar{0.60} \protect\\
        \textbf{Delphi} & \progressbar{0.25} \protect\\
        \textbf{JavaScript} & \progressbar{0.30} \protect\\
        \textbf{Java} & \progressbar{0.50} \protect\\
        \textbf{Lua} & \progressbar{0.55} \protect\\
        \textbf{Python} & \progressbar{0.15} \protect\\
        \textbf{TypeScript} & \progressbar{0.30}
    \end{tabular}
}

\vspace{2.5mm}

\cvitem{\textbf{Databases}}{MySQL, PostgreSQL}

\vspace{2.5mm}

\cvitem{\textbf{Source control}}
{
    \begin{tabular}{p{25mm}p{\linewidth}}
        \textbf{Git} & \progressbar{0.80} \protect\\
        \textbf{SourceSafe} & \progressbar{0.70} \protect\\
        \textbf{SVN} & \progressbar{0.25}
    \end{tabular}
}

\vspace{5mm}

\cvitem{-}{Used to the routines of different project management approaches;}
\cvitem{-}{Familiar with embedded APIs of parallelism, PThreads, OpenMP and MPI.}

\CF

\end{document}
