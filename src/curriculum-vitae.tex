%% Project: Curriculum-Vitae
 % -------------------------------------
 %
 % Author:            Maurício Spinardi
 % Creation date:     2019-01-14
 % File:              curriculum-vitae.tex
 % License:           <to be filled>
 % Plataform:         <not especified>
 % 
 % Description: <to be filled>
 % 
 % -----------------------------------------------------------------------------

% ************************************ %
% Document class and definitions       %
% ************************************ %

% Document class options are: ('10pt', '11pt' and '12pt'), ('a4paper',
% 'letterpaper', 'a5paper', 'legalpaper', 'executivepaper' and 'landscape'),
% ('sans' and 'roman')
\documentclass[11pt,a4paper,sans]{moderncv}

% Fixed style
\moderncvstyle{classic}

% Color options are: 'blue' (default), 'orange', 'green', 'red', 'purple',
% 'grey' and 'black'
\moderncvcolor{blue}

% Family font options are: '\sfdefault' for the default sans serif font,
% '\rmdefault' for the default roman one, or any tex font name
\renewcommand{\familydefault}{\sfdefault}

% Page numbering supressor
%\nopagenumbers{}

% Character encoding
\usepackage[utf8]{inputenc}

% Page margin
\usepackage[scale=0.75]{geometry}

% Modern timeline importing
\usepackage{moderntimeline}

% Timeline horizontal size
\setlength{\hintscolumnwidth}{2.75cm}

% Timeline width
\tlwidth{0}

% Timeline scale
\tlmaxdates{2010}{2019}

% Progress bar
\usepackage{progressbar}

% ModernCV progress bar color
\definecolor{pbblue}{rgb}{0.22,0.45,0.70}

% Booktabs
\usepackage{booktabs}

% C++
\usepackage{relsize}
\newcommand\CC{C\nolinebreak[4]\hspace{-.05em}\raisebox{.4ex}{\relsize{-3}{\textbf{++}}}}

% ************************************ %
% Personal data                        %
% ************************************ %

% Mandatory data
% --------------------------------------

\name{\Huge Maurício}{Spinardi}

% Optional data
% --------------------------------------

\title{\Large Mestrando em Computação \protect\\Científica e Ciência Computacional}

\address{Rua Ten-Cel Luiz Antônio Anhaia, 73}{Vila Gardiman}{13309-391, Itu, SP, BR}

% Phone options are: 'mobile', 'fixed' and 'fax'
\phone[mobile]{+55~(11)~98582~1184~[TIM]}

\email{mauricio.spinardi@gmail.com}

%\homepage{www.johndoe.com}

\extrainfo{26 anos, solteiro}

\photo[64pt][0.4pt]{../res/profile_pic}

%\quote{Some quote (optional)}

% Bibliography
% --------------------------------------

\makeatletter
\renewcommand*{\bibliographyitemlabel}{\@biblabel{\arabic{enumiv}}}
\makeatother

\usepackage{multibib}
%\newcites{book,misc}{{Livros},{Outros}}

% ************************************ %
% Content                              %
% ************************************ %

\begin{document}

% Header
% --------------------------------------

\maketitle

% 
% --------------------------------------
\section{Formação acadêmica}

\tlcventry{2017}{0}{Mestrado em Computação Científica e Ciência Computacional}
{Programa de Pós-graduação em Ciência da Computação, UFSCar}
{Sorocaba}{SP}{Previsão de encerramento: 2019}

\tlcventry{2011}{2016}{Bacharelado em Ciência da Computação}
{UFSCar}
{Sorocaba, SP}{}{}

\tllabelcventry{2010}{2010}{2010}{Ensino Médio}
{Colégio Terras}
{Itu}{SP}{}

% 
% --------------------------------------

\section{Resumo}

\cvitem{}{Desenvolvedor de aplicações embarcadas, com experiência em C, \CC,
Java e Lua, meios de captura (mensageria ISO 8583, processamento EMV) e
manutenção reativa. Mestrando pela Universidade Federal de São Carlos,
pesquisando novas abordagens para Seleção de Fornecedores no contexto
\textit{B2B} da Indústria 4.0.}

% 
% --------------------------------------

\section{Experiência profissional}

\tldatecventry{2017}{\textit{\small{(Atual)}} Analista de Sistemas Júnior}
{SETIS Automação e Sistemas Ltda}{São Paulo}{}
{Desenvolvedor de aplicações embarcadas para terminais POS e outras soluções de
pagamento eletrônico com cartão.}

\tldatecventry{2016}{Estagiário}
{SETIS Automação e Sistemas Ltda}{São Paulo}{}
{Oportunidade de aprendizado e iniciação profissional no setor de pagamento
eletrônico com cartão.}

\tldatecventry{2014}{Monitor em Programação Orientada a Objetos}
{UFSCar}{Sorocaba}{}
{Suporte na formação de colegas de profissão durante o aprendizado de
Programação Orientada a Objetos.}

% 
% --------------------------------------

\section{Idiomas}

\cvdoubleitem{Português}{\textbf{Nativo}}{English}{\textbf{Intermediate \tiny{TOEFL®ITP: 613/677}}}
\cvdoubleitem{Español}{\textbf{Intermedio}}{}{}

\vfill
Desenvolvido em \LaTeX

% 
% --------------------------------------

\newpage

\section{Certificações e Eventos}

\begin{minipage}[htb]{.5\textwidth}
    \cvitem{título}{\emph{Telium SDK Training}}
    \cvitem{ministro(a)}{Denise Tanizaka}
    \cvitem{instituição}{Ingenico Group}
    \cvitem{data}{novembro de 2017}
\end{minipage}
\begin{minipage}[htb]{.5\textwidth}
    \cvitem{título}{\emph{Restrição = Inovação}}
    \cvitem{ministro(a)}{Fabio Akita}
    \cvitem{instituição}{Universidade Federal de São Carlos}
    \cvitem{data}{agosto de 2015}
\end{minipage}

\vspace{5mm}

\begin{minipage}[htb]{.5\textwidth}
    \cvitem{título}{\emph{Free Software and Your Freedom}}
    \cvitem{ministro(a)}{Richard M. Stallman}
    \cvitem{instituição}{Universidade Federal de São Carlos}
    \cvitem{data}{janeiro de 2013}
\end{minipage}

% 
% --------------------------------------

\section{Habilidades}

% About development skills
\subsection{Desenvolvimento}

\cvitem{\textbf{Linguagens}}{C, \CC, Delphi, Java, Lua, Python}

\vspace{2.5mm}

\cvitem{}
{
    \begin{tabular}{p{25mm}p{\linewidth}}
        \textbf{C}              & \progressbar[filledcolor=pbblue,heightr=1,roundnessa=2pt,width=.75\linewidth]{0.90} \\
        \textbf{\CC}            & \progressbar[filledcolor=pbblue,heightr=1,roundnessa=2pt,width=.75\linewidth]{0.60} \\
        \textbf{Delphi}         & \progressbar[filledcolor=pbblue,heightr=1,roundnessa=2pt,width=.75\linewidth]{0.45} \\
        \textbf{Java}           & \progressbar[filledcolor=pbblue,heightr=1,roundnessa=2pt,width=.75\linewidth]{0.70} \\
        \textbf{Lua}            & \progressbar[filledcolor=pbblue,heightr=1,roundnessa=2pt,width=.75\linewidth]{0.80} \\
        \textbf{Python}         & \progressbar[filledcolor=pbblue,heightr=1,roundnessa=2pt,width=.75\linewidth]{0.35}
    \end{tabular}
}

\vspace{2.5mm}

\cvitem{\textbf{Bases de dados}}{MySQL, PostgreSQL}
\cvitem{\textbf{Metodologias}}{SCRUM, Reativa}
\cvitem{\textbf{Versionadores}}{Git, SourceSafe, SVN}

\vspace{2.5mm}

\cvitem{}
{
    \begin{tabular}{p{25mm}p{\linewidth}}
        \textbf{Git}            & \progressbar[filledcolor=pbblue,heightr=1,roundnessa=2pt,width=.75\linewidth]{0.85} \\
        \textbf{SourceSafe}     & \progressbar[filledcolor=pbblue,heightr=1,roundnessa=2pt,width=.75\linewidth]{0.75} \\
        \textbf{SVN}            & \progressbar[filledcolor=pbblue,heightr=1,roundnessa=2pt,width=.75\linewidth]{0.35}
    \end{tabular}
}

\vspace{5mm}

% About other skills
\subsection{Outras linhas e tópicos}

\cvitem{}{\LaTeX}
\cvitem{}{SDKs para dispositivos Ingenico (Telium 2/Plus) e Verifone Verix eVo}
\cvitem{}{Sistemas de autorização com ou sem mensageria ISO 8583}
\cvitem{}{Web REST e JSON}
\cvitem{}{Paralelismo e concorrência com PThreads, OpenMP, MPI e APIs embarcadas}

% 
% --------------------------------------

\vfill
Desenvolvido em \LaTeX

\end{document}
