\iffalse
 * Author:          Maurício Spinardi
 * Creation date:   2019-01-14
 * File:            curriculum-vitae.tex
 * License:         LPPL 1.3c
 * Platform:        <not specified>
 * Project:         curriculum-vitae
 *
 * Description: Curriculum Vitae [pt-br]
\fi

% Document class options are: ('10pt', '11pt' and '12pt'), ('a4paper',
% 'letterpaper', 'a5paper', 'legalpaper', 'executivepaper' and 'landscape'),
% ('sans' and 'roman')
\documentclass[11pt,a4paper,sans]{moderncv}

\usepackage[scale=0.75]{geometry}
\usepackage[utf8]{inputenc}
\usepackage{booktabs}
\usepackage{moderntimeline}
\usepackage{progressbar}
\usepackage{relsize}

% Document settings
% --------------------------------------

\moderncvcolor{blue}
\moderncvstyle{classic}
\renewcommand{\familydefault}{\sfdefault}

% Timeline settings
% --------------------------------------

\setlength{\hintscolumnwidth}{2.85cm}       % Length
\tlmaxdates{2011}{2019}                     % Scale
\tlwidth{0}

% Progress bar definitions
% --------------------------------------

\definecolor{pbblue}{rgb}{0.22,0.45,0.70}
\progressbarchange{filledcolor=pbblue,heightr=1,roundnessa=2pt,
width=.75\linewidth}

% Customized commands
% --------------------------------------

\newcommand\CC{C\nolinebreak[4]\hspace{-.05em}\raisebox{.4ex}{\relsize{-3}
{\textbf{++}}}}

% Profile header
% --------------------------------------

\name{\Huge Maurício}{Spinardi}
\title{\Large Desenvolvedor de aplicações embarcadas e soluções de pagamento
eletrônico}
\address{Rua Glicério, 301}{Liberdade}{01514-000, São Paulo, SP, Brasil}
\phone[mobile]{+55~(11)~98582~1184~[TIM]}
\email{mauricio.spinardi@gmail.com}
\extrainfo{26 anos, solteiro}
\photo[64pt][0.4pt]{../misc/profile_pic}

% Document
% --------------------------------------

\begin{document}

\maketitle

\section{Formação acadêmica}
% --------------------------------------

\tlcventry{2017}{2019}
{Mestrado em Computação Científica e Ciência Computacional}
{Programa de Pós-graduação em Ciência da Computação, UFSCar}
{Sorocaba}{SP}{}

\tlcventry{2011}{2016}
{Bacharelado em Ciência da Computação}
{UFSCar}
{Sorocaba}{SP}{}

\section{Resumo}
% --------------------------------------

\cvitem{}{Experiente em C, C++, Java e Lua, meios de captura (mensageria ISO
8583, processamento EMV) e manutenção reativa. Familiar com as plataformas
Verifone Verix eVo/V-OS2 e Ingenico Telium 2/Telium Plus (Tetra).}

\section{Experiência profissional}
% --------------------------------------

\tlcventry{2017}{0}
{\textit{\small{(Atual)}} Analista de Sistemas Júnior}
{SETIS Automação e Sistemas Ltda}{São Paulo}{}
{Desenvolvedor de soluções embarcadas de pagamento eletrônico com cartão em
terminais POS e PINPADs.}

\tldatecventry{2016}
{Estagiário}
{SETIS Automação e Sistemas Ltda}{São Paulo}{}
{Oportunidade de aprendizado e iniciação profissional no setor de pagamento
eletrônico com cartão.}

\tldatecventry{2014}
{Monitor em Programação Orientada a Objetos}
{UFSCar}{Sorocaba}{}
{Suporte na formação de colegas de profissão durante o aprendizado de
Programação Orientada a Objetos.}

\section{Idiomas}
% --------------------------------------

\cvitem{Português}{\textbf{Nativo}}
\cvitem{English}{\textbf{Intermediate \tiny{TOEFL®ITP: 613/677 (Exp. date: Oct.
2019)}}}
\cvitem{Español}{\textbf{Intermedio}}

\vfill
\textcolor{gray}{Desenvolvido em \LaTeX \protect \\
\tiny{\url{https://github.com/spinardi/curriculum-vitae}}}

% ------------------------------------------------------------------------------
\newpage

\section{Certificações e Eventos}
% --------------------------------------

\begin{minipage}[htb]{.5\textwidth}
    \cvitem{título}{\emph{Telium SDK Training}}
    \cvitem{ministro(a)}{Denise Tanizaka}
    \cvitem{instituição}{Ingenico Group}
    \cvitem{data}{novembro de 2017}
\end{minipage}
\begin{minipage}[htb]{.5\textwidth}
    \cvitem{título}{\emph{Restrição = Inovação}}
    \cvitem{ministro(a)}{Fabio Akita}
    \cvitem{instituição}{Universidade Federal de São Carlos}
    \cvitem{data}{agosto de 2015}
\end{minipage}

\vspace{5mm}

\begin{minipage}[htb]{.5\textwidth}
    \cvitem{título}{\emph{Free Software and Your Freedom}}
    \cvitem{ministro(a)}{Richard M. Stallman}
    \cvitem{instituição}{Universidade Federal de São Carlos}
    \cvitem{data}{janeiro de 2013}
\end{minipage}

\section{Conhecimento técnico e habilidades}
% --------------------------------------

% \vspace{2.5mm}

\cvitem{\textbf{Linguagens}}
{
    \begin{tabular}{p{25mm}p{\linewidth}}
        \textbf{C} & \progressbar{0.90} \protect\\
        \textbf{Delphi} & \progressbar{0.30} \protect\\
        \textbf{JavaScript} & \progressbar{0.35} \protect\\
        \textbf{Java} & \progressbar{0.65} \protect\\
        \textbf{Lua} & \progressbar{0.75} \protect\\
        \textbf{Python} & \progressbar{0.25} \protect\\
        \textbf{TypeScript} & \progressbar{0.35} \protect\\
        \textbf{\CC} & \progressbar{0.60}
    \end{tabular}
}

\vspace{2.5mm}

\cvitem{\textbf{Banco de dados}}{MySQL, PostgreSQL}
\cvitem{\textbf{Metodologias}}{Clássica, SCRUM, Kanban}

\vspace{2.5mm}

\cvitem{\textbf{Versionadores}}
{
    \begin{tabular}{p{25mm}p{\linewidth}}
        \textbf{Git} & \progressbar{0.85} \protect\\
        \textbf{SourceSafe} & \progressbar{0.75} \protect\\
        \textbf{SVN} & \progressbar{0.25}
    \end{tabular}
}

\vspace{5mm}

\subsection{Tópicos diversos}

\cvitem{}{Conhecimento básico de Web REST e JSON}
\cvitem{}{Familiar com APIs embarcadas de paralelismo, PThreads, OpenMP e MPI}

\vfill
\textcolor{gray}{Desenvolvido em \LaTeX \protect \\
\tiny{\url{https://github.com/spinardi/curriculum-vitae}}}

\end{document}
