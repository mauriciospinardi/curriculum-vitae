%% Project: Curriculum-Vitae
 % -------------------------------------
 %
 % Author:            Maurício Spinardi
 % Creation date:     2019-01-14
 % File:              curriculum-vitae.tex
 % License:           <to be filled>
 % Plataform:         <not especified>
 % 
 % Description: <to be filled>
 % 
 % -----------------------------------------------------------------------------

% ************************************ %
% Document class and definitions       %
% ************************************ %

% Document class options are: ('10pt', '11pt' and '12pt'), ('a4paper',
% 'letterpaper', 'a5paper', 'legalpaper', 'executivepaper' and 'landscape'),
% ('sans' and 'roman')
\documentclass[11pt,a4paper,sans]{moderncv}

% Fixed style
\moderncvstyle{classic}

% Color options are: 'blue' (default), 'orange', 'green', 'red', 'purple',
% 'grey' and 'black'
\moderncvcolor{blue}

% Family font options are: '\sfdefault' for the default sans serif font,
% '\rmdefault' for the default roman one, or any tex font name
\renewcommand{\familydefault}{\sfdefault}

% Page numbering supressor
%\nopagenumbers{}

% Character encoding
\usepackage[utf8]{inputenc}

% Page margin
\usepackage[scale=0.75]{geometry}

% Modern timeline importing
\usepackage{moderntimeline}

% Timeline horizontal size
\setlength{\hintscolumnwidth}{2.75cm}

% Timeline width
\tlwidth{0}

% Timeline scale
\tlmaxdates{2010}{2019}

% C++
\usepackage{relsize}
\newcommand\CC{C\nolinebreak[4]\hspace{-.05em}\raisebox{.4ex}{\relsize{-3}{\textbf{++}}}}

% ************************************ %
% Personal data                        %
% ************************************ %

% Mandatory data
% --------------------------------------

\name{\Huge Maurício}{Spinardi}

% Optional data
% --------------------------------------

\title{\Large Mestrando em Computação \protect\\Científica e Ciência Computacional}

\address{Rua Ten-Cel Luiz Antônio Anhaia, 73}{Vila Gardiman}{13309-391, Itu, SP, BR}

% Phone options are: 'mobile', 'fixed' and 'fax'
\phone[mobile]{+55~(11)~98582~1184~[TIM]}

\email{mauricio.spinardi@gmail.com}

%\homepage{www.johndoe.com}

\extrainfo{26 anos, solteiro}

\photo[64pt][0.4pt]{../res/profile_pic}

%\quote{Some quote (optional)}

% Bibliography
% --------------------------------------

\makeatletter
\renewcommand*{\bibliographyitemlabel}{\@biblabel{\arabic{enumiv}}}
\makeatother

\usepackage{multibib}
%\newcites{book,misc}{{Livros},{Outros}}

% ************************************ %
% Content                              %
% ************************************ %

\begin{document}

% Header
% --------------------------------------

\maketitle

% 
% --------------------------------------
\section{Formação acadêmica}

\tlcventry{2017}{0}{Mestrado em Computação Científica e Ciência Computacional}
{Programa de Pós-graduação em Ciência da Computação, UFSCar}
{Sorocaba}{SP}{Previsão de encerramento: 2019}

\tlcventry{2011}{2016}{Bacharelado em Ciência da Computação}
{UFSCar}
{Sorocaba, SP}{}{}

\tllabelcventry{2010}{2010}{2010}{Ensino Médio}
{Colégio Terras}
{Itu}{SP}{}

% 
% --------------------------------------

\section{Resumo}

\cvitem{}{Desenvolvedor de aplicações embarcadas, com experiência em C, \CC,
Java e Lua, meios de captura (mensageria ISO 8583, processamento EMV) e
manutenção reativa. Mestrando pela Universidade Federal de São Carlos, em
Computação Científica e Inteligência Computacional.}

% 
% --------------------------------------

\section{Experiência profissional}

\tldatecventry{2017}{\textit{(Atual)} Analista de Sistemas Júnior}
{SETIS Automação e Sistemas Ltda}{São Paulo}{}
{Desenvolvedor de aplicações embarcadas para terminais POS e outras soluções de
pagamento eletrônico com cartão.}

\tldatecventry{2016}{Estagiário}
{SETIS Automação e Sistemas Ltda}{São Paulo}{}
{Oportunidade de aprendizado e iniciação profissional no setor de pagamento
eletrônico com cartão.}

\tldatecventry{2014}{Monitor em Programação Orientada a Objetos}
{UFSCar}{Sorocaba}{}
{Suporte na formação de colegas de profissão durante o processo de aprendizado
de Programação Orientada a Objetos. Orientado pela prof.ª Dr.ª Katti Faceli,
ganhadora do prêmio Jabuti na categoria Tecnologia e Informática.}

% 
% --------------------------------------

\vfill
Desenvolvido em \LaTeX

\end{document}
