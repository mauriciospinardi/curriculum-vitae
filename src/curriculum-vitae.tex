%
% @file curriculum-vitae.tex
% @author Maurício Spinardi
% @platform N/A
% @brief Curriculum Vitae [pt-br]
% @date 2019-01-14
%

% Document class options are: ('10pt', '11pt' and '12pt'), ('a4paper',
% 'letterpaper', 'a5paper', 'legalpaper', 'executivepaper' and 'landscape'),
% ('sans' and 'roman')
\documentclass[11pt,a4paper,sans]{moderncv}

\usepackage[bottom=3cm,scale=0.75,top=3cm]{geometry}
\usepackage[utf8]{inputenc}
\usepackage{booktabs}
\usepackage{progressbar}
\usepackage{relsize}

% Document settings
% --------------------------------------

\moderncvcolor{blue}
\moderncvstyle{classic}

% Progress bar definitions
% --------------------------------------

\definecolor{pbblue}{rgb}{0.22,0.45,0.70}
\progressbarchange{filledcolor=pbblue,heightr=1,roundnessa=2pt,width=.75\linewidth}

% Customized commands
% --------------------------------------

\newcommand\CPP{C\nolinebreak[4]\hspace{-.05em}\raisebox{.4ex}{\relsize{-3}{\textbf{++}}}}

\newcommand\CF{\vfill\begin{flushright}\textcolor{gray}{Desenvolvido em \LaTeX\protect\\\small{\url{https://github.com/mauriciospinardi}}}\end{flushright}}

% Profile header
% --------------------------------------

\name{\Huge Maurício}{Spinardi}
\title{\Large Desenvolvedor de soluções\protect\\embarcadas de pagamento eletrônico}
\address{}{Vila Gardiman, 13309-391}{Itu, SP, Brasil}
\phone[mobile]{+55~(11)~98582~1184~[VIVO]}
\email{mauricio.spinardi@gmail.com}
\extrainfo{28 anos, solteiro}
\photo[64pt][0.4pt]{../misc/profile_pic}

% Document
% --------------------------------------

\begin{document}

\maketitle

\section{Resumo}
% --------------------------------------

\vspace{0.5mm}

\hspace{1.25cm} Experiente em C, C++, Java e Lua, meios de captura (mensageria
ISO 8583, processamento EMV com/sem contato) e manutenção reativa. Familiar com
equipamentos Verifone e Ingenico.\protect\\

\hspace{1.25cm} Destaques como desenvolvedor: \textit{POS PagBem, POS SafraPay,
POS Rede, PINPAD ABECS, PINPAD Listo, PINPAD Bradesco e Biblioteca
Compartilhada (BC).}

\hspace{1.25cm} Destaques como Engenheiro de Suporte ou Consultor Especialista
Verifone: \textit{POS Vero Banrisul (3IA), PW-HAL e POSWEB (Muxi), PagSeguro
(UOL), POS UNICA (Tribanco) e Stone.}

\section{Experiência profissional}
% --------------------------------------

\vspace{0.5mm}

\hspace{1.25cm} \textit{CloudWalk, Inc}

\cvitem{desde nov/20}{\textbf{Engenheiro de Software}}

\hspace{1.25cm} \begin{minipage}[htb]{\linewidth - 1.25cm}
    Desenvolvimento e manutenção de soluções de captura CloudWalk, Inc.
\end{minipage}

\vspace{\baselineskip}

\hspace{1.25cm} \textit{Verifone do Brasil Ltda}

\cvitem{abr/17 - out/20}{\textbf{Fornecedor de serviços terceirizado}}

\hspace{1.25cm} \begin{minipage}[htb]{\linewidth - 1.25cm}
    Suporte e consultoria especializada para equipamentos Verifone
    Engage:
    \begin{itemize}
        \item[-] Interface de baixo nível (C/C++)
        \item[-] Criação e manutenção de projetos (VDE, FST e Cygwin)
        \item[-] Controle de versão (Git)
        \item[-] Construção de pacotes (Shell), instalação e atualizações (OTA)
    \end{itemize}
\end{minipage}

\vspace{\baselineskip}

\hspace{1.25cm} \textit{SETIS Automação e Sistemas Ltda}

\cvitem{jan/17 - out/20}{\textbf{Analista de Sistemas}}

\hspace{1.25cm} \begin{minipage}[htb]{\linewidth - 1.25cm}
    Desenvolvimento e manutenção de soluções embarcadas de pagamento
    eletrônico:
    \begin{itemize}
        \item[-] Processamento de cartões EMV
        \item[-] Desenvolvimento multiplataforma (C e Lua)
        \item[-] Análise e implementação de regras de negócio (C/C++, Lua,
                 HTML, XML e JSON)
        \item[-] Interface de baixo nível (C/C++)
    \end{itemize}
\end{minipage}

\vspace{\baselineskip}

\hspace{1.25cm} \textit{SETIS Automação e Sistemas Ltda}

\cvitem{mar/16 - dez/16}{\textbf{Estagiário}}

\vspace{\baselineskip}

\hspace{1.25cm} \textit{Universidade Federal de São Carlos - Campus Sorocaba}

\cvitem{mar/14 - jun/14}{\textbf{Monitor de Programação Orientada a Objetos}}

\CF

% ------------------------------------------------------------------------------
\newpage

\section{Formação acadêmica}
% --------------------------------------

\vspace{0.5mm}

\cvitem{2017-2019}{\textbf{Mestrado em Computação Científica e Ciência Computacional}}
\cvitem{}{\small{Universidade Federal de São Carlos. Sorocaba, São Paulo.}}

\cvitem{2011-2016}{\textbf{Bacharelado em Ciência da Computação}}
\cvitem{}{\small{Universidade Federal de São Carlos. Sorocaba, São Paulo.}}

\section{Idiomas}
% --------------------------------------

\vspace{0.5mm}

\cvitem{Português}{\textbf{Nativo}}
\cvitem{English}{\textbf{Intermediate \tiny{TOEFL®ITP: 613/677 (Exp. date: Oct. 2019)}}}
\cvitem{Español}{\textbf{Intermedio}}

\section{Certificações e Eventos}
% --------------------------------------

\vspace{0.5mm}

\begin{minipage}[htb]{.5\linewidth}
    \cvitem{título}{\emph{Telium SDK Training}}
    \cvitem{ministro(a)}{Denise Tanizaka}
    \cvitem{instituição}{Ingenico Group}
    \cvitem{data}{novembro de 2017}
\end{minipage}
\begin{minipage}[htb]{.5\linewidth}
    \cvitem{título}{\emph{Restrição = Inovação}}
    \cvitem{ministro(a)}{Fabio Akita}
    \cvitem{instituição}{Universidade Federal de São Carlos}
    \cvitem{data}{agosto de 2015}
\end{minipage}

\vspace{5mm}

\begin{minipage}[htb]{.5\linewidth}
    \cvitem{título}{\emph{Free Software and Your Freedom}}
    \cvitem{ministro(a)}{Richard M. Stallman}
    \cvitem{instituição}{Universidade Federal de São Carlos}
    \cvitem{data}{janeiro de 2013}
\end{minipage}

\section{Conhecimento técnico e habilidades}
% --------------------------------------

\vspace{0.5mm}

\hspace{1.25cm} \begin{minipage}[htb]{\linewidth - 1.25cm}
    \begin{itemize}
        \item[-] Habituado às rotinas de diferentes metodologias de trabalho;
        \item[-] Familiar com APIs embarcadas de paralelismo e assincronismo,
                 PThreads, OpenMP e MPI.
    \end{itemize}
\end{minipage}

\vspace{5mm}

\cvitem{\textbf{Linguagens de programação}}
{
    \begin{tabular}{p{25mm}p{\linewidth}}
        \textbf{C/\CPP} & \progressbar{0.90}\protect\\
        \textbf{Delphi} & \progressbar{0.15}\protect\\
        \textbf{JavaScript} & \progressbar{0.25}\protect\\
        \textbf{Java} & \progressbar{0.45}\protect\\
        \textbf{Lua} & \progressbar{0.50}\protect\\
        \textbf{Python} & \progressbar{0.20}\protect\\
        \textbf{TypeScript} & \progressbar{0.25}
    \end{tabular}
}

\vspace{2.5mm}

\cvitem{\textbf{Banco de dados}}{MySQL, PostgreSQL}

\vspace{2.5mm}

\cvitem{\textbf{Versionadores}}
{
    \begin{tabular}{p{25mm}p{\linewidth}}
        \textbf{Git} & \progressbar{0.80}\protect\\
        \textbf{SourceSafe} & \progressbar{0.70}\protect\\
        \textbf{SVN} & \progressbar{0.25}
    \end{tabular}
}

\CF

\end{document}
